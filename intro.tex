\section{緒言}

意味のある部分間の関係を考えるプログラミング学習において,課題外在性負荷を減らすため,カード操作方式による学習支援システム(Card Operation-Based Programming Learning Support System,以降,COPS)が開発されている\cite{matsumoto2018}.大学講義で従来システムを導入した結果,非本質的な認知負荷を減らしながら,教授者が意図した学習活動に集中できていたこと,とりわけ初学者にとって有効な学習方法であることが示唆された.また,従来のシステムは,従来のコーディング主体の学習とは同等の学習効果を有しながら,従来よりも学習時間を短縮できる効率的な学習方法であることが明らかにされた.一方,先行研究の中で,COPSの学習の質を評価するため,学習ログデータを知識工学の考え方に基づき分析したところ,適切な学習活動を行っていない学習者の存在が確認された.具体的には,知識を用いず,フィードバックとして与えられたヒントのみを頼りに網羅的に探索すること(知識無し解法)の合理性が示唆された.このような結果が得られた理由は様々考えられるが,可能性の一つとして,既存のフィードバック機能の設計に原因があったのではないかと先行研究によって指摘された.COPSでは,学習者が回答した際,「プログラムの実行結果」や「正誤のフィードバック」と共に,「カード配置の適切さ」の3つの情報が学習者に通知される.これら3つのうち,「カード配置の適切さ」について,必要以上の手助けを行っていたために,試行錯誤的なカード順列の設置・繰り返しが合理的戦略であった可能性が高く,適切な学習の妨げになっていたと結論付けられていた.ただし,この結論はあくまで推測に過ぎない.そこで本研究では,「カード配置の適切さ」に関するフィードバック方式の適切性を実験的に明らかにすることを目的とする.